\documentclass{article}
\usepackage{graphicx}
\graphicspath{ {./images/} }
\usepackage{subcaption}
\usepackage{multicol}
\usepackage[utf8]{inputenc}
\usepackage[english]{babel}
\usepackage{geometry}
\geometry{margin=1in}
\title{Neural Network Comparison for Time Series Forecasting}
\author{Andy Chen, Boo Fullwood, Yunzhou Liu}
\date{May 05, 2020}

\begin{document}

\maketitle

\begin{multicols}{2}
  
\section*{Introduction}

The COVID-19 (or coronavirus) pandemic has forced the widespread deployment of
lockdowns, stay-at-home orders, and other social distancing measures to try and
contain its spread, while policymakers and health researchers have been
interested in forecasting the spread of the disease in order to help prepare and
combat against it. This final project attempts to use machine learning
techniques, specifically standard feed-forward and LSTM neural networks, to
forecast future coronavirus cases in the United States for certain counties and
across the state level in aggregate, based on data for past cases and deaths.

Additionally, we use the Google Community Mobility dataset to generate
additional features for our neural networks in an attempt to improve forecast
accuracy.

\section*{Related Work}


A LSTM neural network addresses a particular shortcoming of traditional neural
networks: an inability to learn long-term dependencies in the network. As a
result, they are well-suited for modeling time series data, such as the number
of cases per day in an epidemic. The coronavirus pandemic arrived only recently
in the United States, but past work on applying feed-forward and LSTM neural
network models to epidemic forecasting do exist in the literature. Wang et al.
(2019) used multiple models, including LSTM neural networks, to model an HIV
epidemic in Guangxi, China, finding evidence for its effectiveness over other
time-series models. Mussumeci and Coelho (2020) used LSTM in comparison to LASSO
and Random Forest regression to forecast weekly incidence of Dengue fever in
Brazil, again finding evidence for LSTM’s effectiveness. With regards to
forecasting the coronavirus pandemic itself, Tomar and Gupta (2020) used an LSTM
model to forecast coronavirus cases in India with somewhat decent results. Yang
et al. (2020) found that an LSTM model trained on 2003 SARS epidemic data
produced an incidence curve that fit surprisingly well to the real one. In
contrast, Punn et al. (2020) predicted global cases over 10 days using only
cases, deaths, and recovered data with both a standard deep neural network and
an LSTM model, but found in comparison that polynomial regression was superior
in the end. Nevertheless, these studies indicate potential for using an LSTM
neural network for coronavirus cases forecasting.\\

Even non-LSTM models, however, can still demonstrate good prediction accuracy. A
multilayer convolutional neural network using multiple inputs for cases and
deaths forecasted the total number of confirmed cases in various Chinese cities
with decent accuracy (Huang et al., 2020). As such, CNNs would also be good to
try for predicting total cases.\\


\section*{Methodology}

Dataset: We use data on U.S. COVID-19 cases and deaths from the Center for
Systems Science and Engineering at Johns Hopkins University, already organized
by state and county upon download. After exploring several models across
different team members, using cumulative U.S. cases and deaths data for one
set of models and county-level U.S. cases data for the other, we decided that
the following data transformations should be applied for COVID-19 confirmed
cases and deaths when exploring neural network structures:

\begin{enumerate}
    \item Since there are too many counties in the dataset, we cannot create
    one-hot encodings for each of the counties, and thus we tried to encode
    different counties by numbers, and view them as discrete quantitative
    variables.

    \item Create past records for each county. For each past day, the record is
    added as a new column to the dataframe. 

    \item Normalization of the records. For each county, we calculate their
    normalized cumulative confirmed cases by $\frac{x-\mu_{x}}{\sigma{x}}$.
\end{enumerate}




Originally, we also wanted to use Google Community Mobility data, obtained using
Google’s BigQuery API, as additional inputs into our neural networks. Since
social distancing, if executed properly, can greatly affect the rate of spread
of the coronavirus in an area, mobility data showing how much people tend to
visit certain locations would in theory serve as a proxy for the level of social
distancing in an area, and help predict the number of future cases. However, we
ran out of time to properly implement these additional inputs.


\subsection*{Data Structure}

Initial tests were done with five to ten days of lookback on case or death
counts and the mobility data for the day prior to the desired prediction. In
particular, we try to predict everydays’ confirmed cases for each county by
their past 5 (or 10) days cumulative confirmed cases, along with their encoded
location. The model created has two densely connected hidden layers and one
output layer, utilizing ReLU as the activation function. Since there are too
many counties, we only chose counties in ten states, including North Carolina,
New York, California, and Washington, as training sets, and counties in
Illinois, Texas and Nevada as testing sets. And since we also noticed that
though we already took a subset of the entire dataset, running the entire 1000
iterations for learning still is too time consuming, so we decided to take an
alternative, and instruct the model to stop learning if it observes that the
loss cannot be further improved for 15 consecutive iterations. For the error
function, we used root mean-squared error (RMSE) of the predicted cases. Our
predictions were for the next day.

\subsection*{Model Structure}

Overall, four models were created to evaluate both standard feed-forward
and recurrent networks.
The first standard model consists of two fully connected hidden layers with ReLU
as the activation function, with each layer containing 128 units.
The second is a convolutional neural net (CNN) of a 1 dimensional convolution
aggregated by max pooling.
The two recurrent networks depend on the Long Short Term Memory (LSTM) model and
are built on identical layers of 128 LSTM units.
Both a simple single layer and a stacked double layer were tested.

\end{multicols}

\section*{Results}

  \begin{figure}[!htb]
    \begin{subfigure}{.5\textwidth}
      \centering
      \includegraphics[width=.8\linewidth]{CNNLoss}
      \label{fig:sfig1}
    \end{subfigure}
    \begin{subfigure}{.5\textwidth}
      \centering
      \includegraphics[width=.8\linewidth]{CNNPred}
      \label{fig:sfig2}
    \end{subfigure}
    \begin{subfigure}{.5\textwidth}
      \centering
      \includegraphics[width=.8\linewidth]{SLSTMLoss}
      \label{fig:sfig3}
    \end{subfigure}
    \begin{subfigure}{.5\textwidth}
      \centering
      \includegraphics[width=.8\linewidth]{SLSTMPred}
      \label{fig:sfig4}
    \end{subfigure}
    \begin{subfigure}{.5\textwidth}
      \centering
      \includegraphics[width=.8\linewidth]{StLSTMLoss}
      \label{fig:sfig5}
    \end{subfigure}
    \begin{subfigure}{.5\textwidth}
      \centering
      \includegraphics[width=.8\linewidth]{StLSTMPred}
      \label{fig:sfig6}
    \end{subfigure}
    \caption{Model Training and Predictive Performance}
    \label{fig:fig1}
  \end{figure}

  \begin{table}[h!]
  \begin{centering}
  \begin{tabular}{l | r | r }
    
    Model & MSE & MAE \\
    \hline
    CNN & 0.042 & 0.002 \\
    
    LSTM (Single) & 0.036 & 0.002 \\
    
    LSTM (Stacked) & 0.031 & 0.001
                               
  \end{tabular}
  \caption{Model Loss and Error Values}
  \label{table:1}
  \end{centering}
\end{table}

  \begin{multicols}{2}
    
\subsection*{Basic Neural Net}

\subsection*{Convolutional Neural Net}

The CNN model performed significantly better than expected for a simple model.
It achieved comparable test loss results as the more complex LSTM models below
and produced reasonable predictions out nearly 20 days from the last trained
data point. The model was mostly linear, which matched the nature of the
training data well, though it did diverge slightly from the test data as the
result of moderate non-linear fluctuations.

\subsection*{LSTM Nerual Net}

The single layer LSTM model achieved moderate predictive performance.
With data scaled to the range $[0, 1]$, the models predictive power declined
more rapidly from the last trained data point than the other models.
This resulted in an overall lower performance on the test data.
The stacked LSTM model gave markedly better performance,
maintaining low error even at high distances from the training point.
However, both LSTM models demonstrated an interesting pattern in that both
produced a model that decreases in growth rate as the days progress.
This may be the influence of the memory capability of the LSTM layers,
as the CNN did not produce this effect. It is this curve that primarily
contributes to the single layer LSTM model's difficulty maintaining accuracy.

\subsection*{Extensions}

Although predicting the number of cases on the next day can still be useful, a
more relevant metric to train our model on would be the accuracy of the
predictions for multiple days ahead.

Additionally, major limitations exist on the in-person testing that generated
the coronavirus cases data used, as different states and counties may have had
different levels of access to testing kits or laboratories to process test
results, or may have had differences in how or when cases were reported. The
Google Mobility dataset also was collected from users that opted into a certain
program, restricting how representative it may be. The hope of this project was
to nevertheless produce a neural network model that could predict future
coronavirus cases with some degree of accuracy in spite of the noise in the
data.


\begin{thebibliography}{7}

\bibitem{}
Ayyoubzadeh, S. M., Ayyoubzadeh, S. M., Zahedi, H., Ahmadi, M., \& Kalhori, S. R.
N. (2020). Predicting COVID-19 Incidence Through Analysis of Google Trends Data
in Iran: Data Mining and Deep Learning Pilot Study. JMIR Public Health and
Surveillance, 6(2), e18828.

\bibitem{}
Huang, C. J., Chen, Y. H., Ma, Y., \& Kuo, P. H. (2020). Multiple-Input Deep
Convolutional Neural Network Model for COVID-19 Forecasting in China. medRxiv.

\bibitem{}
Mussumeci, E., \& Coelho, F. C. (2020). Machine-learning forecasting for Dengue
epidemics-Comparing LSTM, Random Forest and Lasso regression. medRxiv.

\bibitem{}
Punn, N. S., Sonbhadra, S. K., \& Agarwal, S. (2020). COVID-19 Epidemic Analysis
using Machine Learning and Deep Learning Algorithms. medRxiv.

\bibitem{}
Tomar, A., \& Gupta, N. (2020). Prediction for the spread of COVID-19 in India
and effectiveness of preventive measures. Science of The Total Environment,
138762.

\bibitem{}
Wang, G., Wei, W., Jiang, J., Ning, C., Chen, H., Huang, J., ... \& Lai, J.
(2019). Application of a long short-term memory neural network: a burgeoning
method of deep learning in forecasting HIV incidence in Guangxi, China.
Epidemiology \& Infection, 147.

\bibitem{}
Yang, Z., Zeng, Z., Wang, K., Wong, S. S., Liang, W., Zanin, M., ... \& Liang, J.
(2020). Modified SEIR and AI prediction of the epidemics trend of COVID-19 in
China under public health interventions. Journal of Thoracic Disease, 12(3),
165.

\end{thebibliography}

\end{multicols}

\end{document}

