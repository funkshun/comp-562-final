\documentclass{article}
\title{Forecasting COVID-19 Cases with Neural Network and LSTM Models}
\author{}
\date{May 05, 2020}

\begin{document}

\maketitle
\newpage

\section*{Introduction}

The COVID-19 (or coronavirus) pandemic has forced the widespread deployment of
lockdowns, stay-at-home orders, and other social distancing measures to try and
contain its spread, while policymakers and health researchers have been
interested in forecasting the spread of the disease in order to help prepare and
combat against it. This final project attempts to use machine learning
techniques, specifically standard feed-forward and LSTM neural networks, to
forecast future coronavirus cases in the United States for certain counties and
across the state level in aggregate, based on data for past cases and deaths.


\section*{Related Work}


A LSTM neural network addresses a particular shortcoming of traditional neural
networks: an inability to learn long-term dependencies in the network. As a
result, they are well-suited for modeling time series data, such as the number
of cases per day in an epidemic. The coronavirus pandemic arrived only recently
in the United States, but past work on applying feed-forward and LSTM neural
network models to epidemic forecasting do exist in the literature. Wang et al.
(2019) used multiple models, including LSTM neural networks, to model an HIV
epidemic in Guangxi, China, finding evidence for its effectiveness over other
time-series models. Mussumeci and Coelho (2020) used LSTM in comparison to LASSO
and Random Forest regression to forecast weekly incidence of Dengue fever in
Brazil, again finding evidence for LSTM’s effectiveness. With regards to
forecasting the coronavirus pandemic itself, Tomar and Gupta (2020) used an LSTM
model to forecast coronavirus cases in India with somewhat decent results. Yang
et al. (2020) found that an LSTM model trained on 2003 SARS epidemic data
produced an incidence curve that fit surprisingly well to the real one. In
contrast, Punn et al. (2020) predicted global cases over 10 days using only
cases, deaths, and recovered data with both a standard deep neural network and
an LSTM model, but found in comparison that polynomial regression was superior
in the end. Nevertheless, these studies indicate potential for using an LSTM
neural network for coronavirus cases forecasting.\\

Even non-LSTM models, however, can still demonstrate good prediction accuracy. A
multilayer convolutional neural network using multiple inputs for cases and
deaths forecasted the total number of confirmed cases in various Chinese cities
with decent accuracy (Huang et al., 2020). As such, CNNs would also be good to
try for predicting total cases.\\


\section*{Methodology}

Dataset: We use COVID-19 U.S. cases and deaths data from the Center for Systems
Science and Engineering at Johns Hopkins University, already organized by state
and county upon download. After a few explorations for different models, we
decided that the following data transformations should be applied for COVID-19
confirmed and death cases:

\begin{enumerate}
    \item Since there are too many counties in the dataset, we cannot create
    one-hot encodings for each of the counties, and thus we tried to encode
    different counties by numbers, and view them as discrete quantitative
    variables.

    \item Create past records for each county. For each past day, the record is
    added as a new column to the dataframe. 

    \item Normalization of the records. For each county, we calculate their
    normalized cumulative confirmed cases by $\frac{x-\mu_{x}}{\sigma{x}}$.
\end{enumerate}




Originally, we also wanted to use Google Community Mobility data, obtained using
Google’s BigQuery API, as additional inputs into our neural networks. Since
social distancing, if executed properly, can greatly affect the rate of spread
of the coronavirus in an area, mobility data showing how much people tend to
visit certain locations would in theory serve as a proxy for the level of social
distancing in an area, and help predict the number of future cases. However, we
ran out of time to properly implement these additional inputs.


\subsection*{Data Structure}

Initial tests were done with five to ten days of lookback on case or death
counts and the mobility data for the day prior to the desired prediction. In
particular, we try to predict everydays’ confirmed cases for each county by
their past 5 (or 10) days cumulative confirmed cases, along with their encoded
location. The model created has two densely connected hidden layers and one
output layer, utilizing ReLU as the activation function. Since there are too
many counties, we only chose counties in ten states, including North Carolina,
New York, California, and Washington, as training sets, and counties in
Illinois, Texas and Nevada as testing sets. And since we also noticed that
though we already took a subset of the entire dataset, running the entire 1000
iterations for learning still is too time consuming, so we decided to take an
alternative, and instruct the model to stop learning if it observes that the
loss cannot be further improved for 15 consecutive iterations. For the error
function, we used root mean-squared error (RMSE) of the predicted cases. Our
predictions were for the next day.


\section*{Results}

\subsection*{LSTM Failure}

Several Long Short Term Memory type neural networks were tried to evaluate the
effectiveness of Recurrent Neural Nets for time series forecasting. Despite a
reputation for high accuracies even with relatively simple structures, we were
unable to achieve performance comparable with non-recurrent models. This is
almost certainly a failure in data preparation rather than an indictment of the
model type given its demonstrated success in other similar forecasting
scenarios. 

\subsection*{Extensions}

Although predicting the number of cases on the next day can still be useful, a
more relevant metric to train our model on would be the accuracy of the
predictions for multiple days ahead.

\end{document}

